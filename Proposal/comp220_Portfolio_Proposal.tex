% Please do not change the document class
\documentclass{scrartcl}

% Please do not change these packages
\usepackage[hidelinks]{hyperref}
\usepackage[none]{hyphenat}
\usepackage{setspace}
\doublespace

% You may add additional packages here
\usepackage{amsmath}
\usepackage{graphicx}
\usepackage{wrapfig}
\graphicspath{ {./images/} }

% Please include a clear, concise, and descriptive title
\title{Portfolio Proposal} 

% Please do not change the subtitle
\subtitle{COMP220 Graphics and Simulation Proposal}

% Please put your student number in the author field
\author{1507516}

\begin{document}

\maketitle

\abstract{}

\section{Concept}
For my concept I will create an open procedurally generated low-poly world that the player will be able to explore.

The player will be a light source that will move around the procedural world, lighting up the area around the player.

The player can fire lots of light sources that will light up the path in head.

The player can suck up all the light sources back into the player and then the player will illuminate more.

\section{How it meets the requirements}

This open world simulation will contain multiple Graphical elements to make the world interesting.

\section{What Graphical or Simulation effects will the demo include?}

\subsection{Procedural Generation of terrain}

The map will be procedurally generated using a Perlin noise height map. Furthermore I will try and populate the level with some foliage to make the world slightly more interesting to explore.

\subsection{Ground Bump Mapping}

The ground will have a bump map for the different types of textures on the ground, so that the ground will look bumpy for rock textures and jagged for grass textures etc.

\section{Additional graphics techniques that may be implemented}

\subsection{Ray-Casting}
  
\subsection{Loading of meshes from a standard 3D object file format}



\end{document}
